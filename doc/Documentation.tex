\documentclass[11pt]{article}
%Gummi|065|=)
\usepackage{graphicx,url}
\usepackage[utf8]{inputenc}
\usepackage[brazil]{babel}
\title{\textbf{Trabalho Prático 2}}
\author{Carolina Santos\\
		Thiago Silva}
\date{}
\begin{document}

\maketitle

\section{Introdução}
Neste trabalho desenvolvemos um sistema de mensagens usando o protocolo TCP e orientação a eventos. O sistema foi desenvolvido em três módulos separados, servidor, exibidor e emissor.\\O servidor é o módulo responsável pela intermediação das mensagens trocadas pelos clientes, o exibidor exibe as mensagens recebidas pelo seu emissor, e o emissor envia mensagens para outros clientes.


\section{Decisões de Projeto}
\subsection{Estrutura do Projeto}
O sistema de mensagens foi separado em três módulos: Servidor, emissor e exibidor.

\subsubsection{Servidor}
O servidor fica em src/server.py
\subsubsection{Emissor}
\subsubsection{Exibidor}


\subsection{Mensagens}
Os únicos tipos de mensagens usadas neste projeto são as descritas na descrição do trabalho.
\begin{enumerate}
\item OI (0): A primeira mensagem mandada pelo emissor ou exibidor para o servidor. Ela serve para criar uma conexão entre duas terminações.

\item FLW (1): Este tipo de mensagem serve para acabar conexões, ela pode ser enviada de um emissor para o servidor, de um emissor para um exibidor e de um exibidor para o servidor.
Uma terminação do emissor não significa uma terminação do seu emissor, neste trabalho decidimos deixar os dois módulos do cliente separados nesta ação.

\item MSG(2): Mensagem enviada de um cliente para outro, ela é escrita pelo próprio usuário.

\item OK (3): Mensagem enviada pelo servidor para o cliente que mandou uma MSG. Um OK é retornado pelo servidor somente se a mensagem está bem formada.

\item ERRO (4): O erro tem um papel parecido do nack, uma mensagem de erro é retornada pelo servidor quando um emissor, ou exibidor, pedem por um id já usado.

\item QEM (5): Mensagem enviada por um cliente para o servidor para saber quais outros clientes estão registrados. Recebido um QEM pelo servidor, ele responde um OKQEM com os clientes abertos.

\item OKQEM (6): Esta mensagem é uma resposta do servidor para o cliente que pediu para listar os outros clientes abertos.

\end{enumerate}


\section {Execução}
Para executar o sistema de mensagens, execute primeiro o servidor com "python src/server.py" seguido pelos emissores e exibidores


\section{Conclusão}


\end{document}
